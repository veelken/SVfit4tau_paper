\section{Introduction}
\label{sec:introduction}

The discovery of a Higgs ($\PHiggs$) boson by the ATLAS and CMS experiments at the LHC~\cite{Higgs-Discovery_CMS,Higgs-Discovery_ATLAS}
represents a major step towards our understanding of electroweak symmetry breaking (EWSB) 
and of the mechanism that generates the masses of quarks and leptons, the particles that constitute the ``ordinary'' matter in our universe.
In a combined analysis of the data recorded by ATLAS and CMS, 
the mass of the $\PHiggs$ boson has been measured to be $125.09 \pm 0.24$~\GeV~\cite{HIG-14-042}.
The Standard Model (SM) of particle physics makes precise predictions for all properties of the $\PHiggs$ boson, given its mass.
So far, all properties that have been measured agree with the expectation for a SM $\PHiggs$ boson~\cite{HIG-15-002}.
The rate for the decay of the discovered particle into a pair of $\Pgt$ leptons has been measured recently
and found to be consistent with the SM expectation within the uncertainties of these measurements, 
amounting to $20$--$30\%$ at present~\cite{HIG-13-004,Aad:2015vsa,HIG-15-002,HIG-16-043,ATLAS:2018lur},
One important prediction of the SM yet has to be verified experimentally, however:
the $\PHiggs$ boson self-interaction.

The SM predicts $\PHiggs$ boson self-interactions via trilinear and quadlinear couplings. 
Measurements of the $\PHiggs$ boson self-interactions will ultimately either confirm or falsify that the Brout-Englert-Higgs (BEH) mechanism of the SM is responsible for EWSB. 
The trilinear coupling ($\lambda_{\PHiggs\PHiggs\PHiggs}$) can be determined at the LHC, 
by measuring the rate for $\PHiggs$ boson pair production ($\PHiggs\PHiggs$). 
The measurement is challenging, due to the small signal cross section, 
which results from the destructive interference of two competing production processes, and from sizeable backgrounds. 
The quadlinear coupling is not accessible at the LHC, as the cross section of the corresponding process, 
triple $\PHiggs$ boson production, is too small to be measured, even with the large dataset collected by the end of LHC operation.
The leading order (LO) Feynman diagrams for SM $\PHiggs\PHiggs$ production are shown in Fig.~\ref{fig:FeynmanDiagrams_smHH}.
The cross section for $\PHiggs\PHiggs$ production is small, amounting to about $\sigma = 34$~fb in proton-proton collisions at $\sqrt{s}=13$~\TeV center-of-mass energy,
due to the destructive interference of the two diagrams.
The ``triangle'' diagram shown on the left depends on $\lambda_{\PHiggs\PHiggs\PHiggs}$,
while the ``box'' diagram shown on the right does not.

\begin{figure}[h!]
\setlength{\unitlength}{1mm}
\begin{center}
\begin{picture}(180,34)(0,0)
\put(1.5, 0.0){\mbox{\includegraphics*[height=34mm]
  {figures/feynman_nonresonant_triangle.pdf}}}
\put(81.5, 0.0){\mbox{\includegraphics*[height=34mm]
  {figures/feynman_nonresonant_box.pdf}}}
\end{picture}
\end{center}
\caption{ LO Feynman diagrams for $\PHiggs\PHiggs$ production within the SM.}
\label{fig:FeynmanDiagrams_smHH}
\label{fig:massDistributions}
\end{figure}

Deviations of $\lambda_{\PHiggs\PHiggs\PHiggs}$ from its SM value of $\lambda_{\textrm{SM}}=1$, referred to as anomalous $\PHiggs$ boson self-couplings,
alter the interference between the two diagrams, 
resulting in a change in the $\PHiggs\PHiggs$ production cross section and a change in the distribution of the mass, $m_{\PHiggs\PHiggs}$, of the $\PHiggs$ boson pair.
Regardless of the value of $\lambda_{\PHiggs\PHiggs\PHiggs}$, a broad distribution in $m_{\PHiggs\PHiggs}$ is expected,
motivating the convention to refer to the interference of box and triangle diagram as ``non-resonant'' $\PHiggs\PHiggs$ production.
The shape of the distribution in $m_{\PHiggs\PHiggs}$ provides a handle to determine $\lambda_{\PHiggs\PHiggs\PHiggs}$,
complementary to measuring the $\PHiggs\PHiggs$ production cross section.
Various scenarios beyond the SM feature anomalous $\PHiggs$ boson self-couplings,
for example two-Higgs-doublet models~\cite{Branco:2011iw}, such as the minimal supersymmetric extension of the SM (MSSM)~\cite{Gunion:1989we},
and models with composite $\PHiggs$ bosons~\cite{Grober:2010yv,Contino:2012xk}.
The prospects for improving the sensitivity to determine $\lambda_{\PHiggs\PHiggs\PHiggs}$ by utilising information on the mass of the $\PHiggs$ boson pair
have been studied in events in which the $\PHiggs$ boson pair decays via $\PHiggs\PHiggs \to \PW\PW\PW\PW$, with subsequent decay of the $\PW$ bosons to electrons, muons, or jets,
in Refs.~\cite{Baur:2002rb,Baur:2002qd}.
The information that can be extracted from the distribution in $m_{\PHiggs\PHiggs}$ is limited, however,
by the fact that the distribution in $m_{\PHiggs\PHiggs}$ changes only moderately with $\lambda_{\PHiggs\PHiggs\PHiggs}$.

The rate for $\PHiggs\PHiggs$ production may be enhanced significantly in case an as yet undiscovered heavy resonance $\textrm{X}$ decays into pairs of $\PHiggs$ bosons.
Several models beyond the SM give rise to such decays, 
for example Higgs portal models~\cite{Englert:2011yb,No:2013wsa} and models involving warped extra dimensions~\cite{Randall:1999ee},
as well as two-Higgs-doublet models and models with composite Higgs bosons.
If the lifetime $t$ of the resonance is sufficiently large, $t \gtrsim 10^{-25}\textrm{~s}/m_{\textrm{X}}\textrm{~[100~GeV]}$, 
where $m_{\textrm{X}$ denotes the mass of the resonance, 
the distribution in the mass $m_{\PHiggs\PHiggs}$ of the $\PHiggs$ boson pair is expected to exhibit a narrow peak at $m_{\textrm{X}}$.

In this paper we present an algorithm for the reconstruction of the mass $m_{\PHiggs\PHiggs}$ of the $\PHiggs$ boson pair 
in events in which the $\PHiggs$ boson pair originates from the decay of a heavy resonance $\textrm{X}$ and decays via $\PHiggs\PHiggs \to \Pgt\Pgt\Pgt\Pgt$,
with subsequent decay of the $\Pgt$ leptons into electrons, muons, or hadrons.
The decay of $\PHiggs$ boson pairs to four $\Pgt$ leptons ($\PHiggs\PHiggs \to \Pgt\Pgt\Pgt\Pgt$) has not been discussed in the literature so far.
This decay channel provides a small branching fraction, but also comparatively low backgrounds.
The algorithm achieves a resolution on $m_{\PHiggs\PHiggs}$ that varies between $7$ and $22\%$,
depending on the mass of the resonance.
We expect that the reconstruction of $m_{\PHiggs\PHiggs}$ will further improve the separation of the $\PHiggs\PHiggs \to \Pgt\Pgt\Pgt\Pgt$ signal from backgrounds,
thereby increasing the sensitivity to either find evidence for the presence of such a signal in the LHC data or to set stringent exclusion limits.

The reconstruction of $m_{\PHiggs\PHiggs}$ in $\PHiggs\PHiggs \to \Pgt\Pgt\Pgt\Pgt$ events is based on the formalism,
developed in Ref.~\cite{SVfitMEM}, for treating $\Pgt$ lepton decays in the so-called matrix element (ME) method~\cite{Kondo:1988yd,Kondo:1991dw}.
The algorithm presented in this paper does not employ the full ME treatment,
but is based on a simplified likelihood approach.
The simplified approach is motivated by the studies performed in Ref.~\cite{SVfitMEM}, 
which found that the difference in mass resolution between the approximate likelihood treatment and the full ME formalism 
is small when applied to the task of reconstructing the mass of the $\PHiggs$ boson in events containing single $\PHiggs$ bosons that decay via $\PHiggs \to \Pgt\Pgt$,
while the likelihood approach provides a significant saving in computing time.

We present the algorithm for reconstructing the mass of the $\PHiggs$ boson pair 
in $\PHiggs\PHiggs \to \Pgt\Pgt\Pgt\Pgt$ events in Section~\ref{sec:algorithm}.
In Section~\ref{sec:performance} we study the resolution achieved by the algorithm
in reconstructing $m_{\PHiggs\PHiggs}$ for hypothetic resonances $\textrm{X}$ of different mass.
We conclude the paper with a summary in Section~\ref{sec:summary}.
