\section{Performance}
\label{sec:performance}

The performance of the algorithm is quantified in terms of the resolution achieved in reconstructing $m_{\PHiggs\PHiggs}$.
The resolution is studied using simulated samples of events
in which a heavy resonance $\textrm{X}$ decays into a pair of $\PHiggs$ bosons,
and the $\PHiggs$ bosons subsequently decay to four $\Pgt$ leptons.
Samples are produced for $m_{\textrm{X}} = 300$, $500$, and $800$~\GeV.
We expect the resolution to be similar for resonances of spin $0$ and spin $2$,
but focus on studying resonances of spin $0$ in this paper.
Events are generated for proton-proton collisions at $\sqrt{s} = 13$~\TeV centre-of-mass energy,
using the leading order program MadGraph, in the version MadGraph\_aMCatNLO 2.3.2.2~\cite{MadGraph_aMCatNLO},
with the NNPDF3.0 set of parton distribution functions~\cite{NNPDF1,NNPDF2,NNPDF3}.
Parton shower and hadronization processes are modelled using the generator PYTHIA 8.2~\cite{pythia8} with the tune CUETP8M1~\cite{PYTHIA_CUETP8M1tune_CMS}.
The decays of $\Pgt$ leptons, including polarization effects, are modelled by PYTHIA.

We select events in the decay channel $\textrm{X} \to \PHiggs\PHiggs \to \Pgt\Pgt\Pgt\Pgt \to \Plepton\Plepton\tauh\tauh$
and study them on generator level.
Reconstruction effects are simulated by varying the generator-level quantities within their experimental resolution,
which we perform by randomly sampling from the TF 
$W_{\Phadron}( \pT^{\vis} | \pThat^{\vis} )$
and
$W_{\rec}( \pX^{\rec},\pY^{\rec} | \pXhat^{\rec},\pYhat^{\rec} )$
described in Section~\ref{sec:algorithm}.
The electrons, muons, and $\tauh$ are required to satisfy conditions on $\pT$ and $\eta$, 
which are typical for data analyses performed by the ATLAS and CMS collaborations during LHC Run $2$.
Electrons (muons) are required to be within $\vert\eta\vert < 2.5$ ($\vert\eta\vert < 2.4$).
The lepton of higher (lower) $\pT$ is required to pass a $\pT$ threshold of $25$ ($15$)~\GeV.
Each of the two $\tauh$ is required to satisfy $\pT > 20$~\GeV and $\vert\eta\vert < 2.3$.

The resolution on $m_{\PHiggs\PHiggs}$ is studied in terms of the ratio between the reconstructed value of $m_{\PHiggs\PHiggs}$ 
and the true mass $m_{\PHiggs\PHiggs}^{\textrm{true}}$ of the $\PHiggs$ boson pair.
Distributions in this ratio are shown in Fig.~\ref{fig:massDistributions}.
They are shown separately for chosen 
(pairings that maximize the likelihood function $\mathcal{P}$) 
and for discarded (other) pairings and for
events in which electrons, muons, and $\tauh$ are correctly associated to $\PHiggs$ boson pairs and events with spurious pairings.
The correct pairing is chosen in $87$, $98$, and $>99\%$ of the events with $m_{\textrm{X}} = 300$, $500$, and $800$~\GeV, respectively.
The resolution on $m_{\PHiggs\PHiggs}$ for the chosen pairings amounts to $22$, $7$, and $9\%$,
relative to the true mass of the $\PHiggs$ boson pair.
The mass resolution for resonances of $m_{\textrm{X}} = 300$, near the kinematic threshold $m_{\textrm{X}} \approx 2 \, m_{\PHiggs}$,
is limited by the fact that the wrong pairing is chosen in $13\%$ of events.
For events in which the correct pairing is chosen, the resolution on $m_{\PHiggs\PHiggs}$ amounts to $4$, $6$, and $8\%$
for $m_{\textrm{X}} = 300$, $500$, and $800$~\GeV, respectively.
We leave the optimization of the choice of the correct pairing for resonances of low mass to future studies.

\begin{figure}
\setlength{\unitlength}{1mm}
\begin{center}
\begin{picture}(180,212)(0,0)
\put(-4.5, 152.0){\mbox{\includegraphics*[height=60mm]
  {plots/makeSVfit4tauPlots_x_to_hh_300_chosen_log.pdf}}}
\put(81.5, 152.0){\mbox{\includegraphics*[height=60mm]
  {plots/makeSVfit4tauPlots_x_to_hh_300_discarded_log.pdf}}}
\put(-4.5, 78.0){\mbox{\includegraphics*[height=60mm]
  {plots/makeSVfit4tauPlots_x_to_hh_500_chosen_log.pdf}}}
\put(81.5, 78.0){\mbox{\includegraphics*[height=60mm]
  {plots/makeSVfit4tauPlots_x_to_hh_500_discarded_log.pdf}}}
\put(-4.5, 4.0){\mbox{\includegraphics*[height=60mm]
  {plots/makeSVfit4tauPlots_x_to_hh_800_chosen_log.pdf}}}
\put(81.5, 4.0){\mbox{\includegraphics*[height=60mm]
  {plots/makeSVfit4tauPlots_x_to_hh_800_discarded_log.pdf}}}
\put(38.0, 148.0){\small (a)}
\put(124.0, 148.0){\small (b)}
\put(38.0, 74.0){\small (c)}
\put(124.0, 74.0){\small (d)}
\put(38.0, 0.0){\small (e)}
\put(124.0, 0.0){\small (f)}
\end{picture}
\end{center}
\caption{
  Distributions in $m_{\PHiggs\PHiggs}$, relative to the true mass $m_{\PHiggs\PHiggs}^{\textrm{true}} = m_{\textrm{X}}$ of the $\PHiggs$ boson pair,
  in events in which a heavy resonance $\textrm{X}$ of mass $m_{\textrm{X}} = 300$ (a,b), $500$ (c,d), and $800$~\GeV (e,f)
  decays via $\textrm{X} \to \PHiggs\PHiggs \to \Pgt\Pgt\Pgt\Pgt \to \Plepton\Plepton\tauh\tauh$.
  The distributions are shown separately for the chosen (a,c,e) and for the discarded (b,d,f) pairings,
  and are further subdivided into correct and spurious pairings.
  The axis of abscissae ranges from $0.2$ to $5$.
}
\label{fig:massDistributions}
\end{figure}

The algorithm requires typically $2$s of CPU time per event to reconstruct $m_{\PHiggs\PHiggs}$.
