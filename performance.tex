\section{Performance}
\label{sec:performance}

The performance of the algorithm is quantified in terms of its resolution on $m_{\PHiggs\PHiggs}$.
The resolution on $m_{\PHiggs\PHiggs}$ is studied using simulated samples of events
in which a heavy resonance $\textrm{X}$ of mass $m_{\textrm{X}} = 300$, $500$, and $800$~\GeV
decays via $\PHiggs\PHiggs \to \Pgt^{+}\Pgt^{-}\Pgt^{+}\Pgt^{-}$.
We expect the resolution to be similar for resonance of spin $0$ and spin $2$,
but focus on studying resonances of spin $0$ in this paper.
Events are generated for proton-proton collisions at $\sqrt{s} = 13$~\TeV centre-of-mass energy,
using the leading order program MadGraph, in the version MadGraph\_aMCatNLO 2.2.2~\cite{MadGraph_aMCatNLO},
with the NNPDF3.0 set of parton distribution functions~\cite{NNPDF1,NNPDF2,NNPDF3}.
Parton shower and hadronization processes are modelled using the generator PYTHIA 8.2~\cite{pythia8} with the tune CUETP8M1~\cite{PYTHIA_CUETP8M1tune_CMS}.
The decays of $\Pgt$ leptons, including polarization effects, are modelled by PYTHIA.

We study the events are studied on generator level.
Reconstruction effects are simulated by randomly varying the generator-level quantities within their experimental resolution.
We use the TF given in Ref.~\cite{SVfitMEM} to model these resolutions.
The resolution on the $\pT$ of $\tauh$ is modeled by the function:
\begin{equation}
W_{\Phadron}( \pT^{\vis} | \pThat^{\vis} ) = 
 \begin{cases}
   \mathcal{N} \, \xi_{1} \, \left( \frac{\alpha_{1}}{x_{1}} - x_{1} - \frac{x - \mu}{\sigma} \right)^{-\alpha_{1}} \,  
 & \mbox{if } x < x_{1} \\
   \mathcal{N} \, \exp\left( -\frac{1}{2} \, \left( \frac{x - \mu}{\sigma} \right)^{2} \right) \,
 & \mbox{if } x_{1} \leq x \leq x_{2} \\
   \mathcal{N} \, \xi_{2} \, \left( \frac{\alpha_{2}}{x_{2}} - x_{2} - \frac{x - \mu}{\sigma} \right)^{-\alpha_{2}} \,
 & \mbox{if } x > x_{2} 
 \end{cases}
\label{eq:tf_tauToHadDecays_pT}
\end{equation}
using the values $\mu = 1.0$, $\sigma = 0.03$, $x_{1} = 0.97$, $\alpha_{1} = 7$,
$x_{2} = 1.03$, and $\alpha_{2} = 3.5$ for its parameters,
while the $\eta$, $\phi$, and $m_{\vis}$ of $\tauh$ are assumed to be reconstructed with negligible experimental resolution.
The latter assumption is also made for the $\pT$, $\eta$, and $\phi$ of electrons and muons.
The momentum of the hadronic recoil is assumed to be reconstructed with a resolution of $10$~\GeV on its components $\pX$ and $\pY$.

Events are selected in the decay channel $\Pgt^{+}\Pgt^{-}\Pgt^{+}\Pgt^{-} \to 2\Plepton + 2\tauh$, 
where the symbol $\Plepton$ refers to an electron or a muon.
Electrons, muons, and $\tauh$ are required to satisfy conditions on $\pT$ and $\eta$, 
which are typical for data analyses performed by the ATLAS and CMS collaborations during LHC Run $2$.
Electrons (muons) are required to be within $\vert\eta\vert < 2.5$ ($\vert\eta\vert < 2.4$).
The lepton of higher (lower) $\pT$ is required to pass a $\pT$ threshold of $25$ ($15$)~\GeV.
Each of the two $\tauh$ is required to satisfy $\pT > 20$~\GeV and $\vert\eta\vert < 2.3$.

-------- BIS HIER !!!

Comment on plots
- mention fraction of signal events for which the correct/wrong pairing was chosen by the criterion of maximing the integrand in Eq.~\ref{eq:cSVfit_with_hadRecoil}

The algorithm requires about $0.5$s of CPU time per event to reconstruct the mass $m_{\PHiggs\PHiggs}$ of the $\PHiggs$ boson pair.
