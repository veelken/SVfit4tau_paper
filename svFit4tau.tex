\def\verPreprint{1}
\def\verPAPER{2}
\def\ver{1}

\ifx\ver\verPreprint
\documentclass[a4paper,english,11pt]{article}
\usepackage[bindingoffset=0.5cm,left=2.5cm,right=2.5cm,top=2.5cm,bottom=2.5cm,footskip=1.0cm]{geometry}
\usepackage{lineno,hepnames,bm,multirow,amssymb,authblk,graphicx,newclude,xspace,hyperref}
\fi
\ifx\ver\verPAPER
\documentclass[1p]{elsarticle}
\usepackage{lineno,hyperref,hepnames,bm,multirow,amssymb,xspace}
\fi

\modulolinenumbers[5]

%%\journal{Journal of \LaTeX\ Templates}

%%%%%%%%%%%%%%%%%%%%%%%
%% Elsevier bibliography styles
%%%%%%%%%%%%%%%%%%%%%%%
%% To change the style, put a % in front of the second line of the current style and
%% remove the % from the second line of the style you would like to use.
%%%%%%%%%%%%%%%%%%%%%%%

%% Numbered
%\bibliographystyle{model1-num-names}

%% Numbered without titles
%\bibliographystyle{model1a-num-names}

%% Harvard
%\bibliographystyle{model2-names.bst}\biboptions{authoryear}

%% Vancouver numbered
%\usepackage{numcompress}\bibliographystyle{model3-num-names}

%% Vancouver name/year
%\usepackage{numcompress}\bibliographystyle{model4-names}\biboptions{authoryear}

%% APA style
%\bibliographystyle{model5-names}\biboptions{authoryear}

%% AMA style
%\usepackage{numcompress}\bibliographystyle{model6-num-names}

%% `Elsevier LaTeX' style
\bibliographystyle{elsarticle-num}
%%%%%%%%%%%%%%%%%%%%%%%

%%%%%%%%%%%%%%%%%%%%%%%
%% Custom latex macros
%%%%%%%%%%%%%%%%%%%%%%%

\renewcommand{\Plepton}{\ensuremath{\ell}}
\newcommand{\Phadron}{\ensuremath{\textrm{h}}}
\newcommand{\tauh}{\ensuremath{\Pgt_{\textrm{h}}}\xspace}
\newcommand{\tauhnu}{\ensuremath{\tauh \, \Pnu_{\kern-0.10em \Pgt}}\xspace}
\newcommand{\tauhnuOne}{\ensuremath{\tauh^{(1)} \, \Pnu^{(1)}_{\kern-0.10em \Pgt}}\xspace}
\newcommand{\tauhnuTwo}{\ensuremath{\tauh^{(2)} \, \Pnu^{(2)}_{\kern-0.10em \Pgt}}\xspace}
\newcommand{\ellnunu}{\ensuremath{\Plepton \, \APnu_{\kern-0.10em \Plepton} \, \Pnu_{\kern-0.10em \Pgt}}\xspace}
\newcommand{\ellnunuOne}{\ensuremath{\Plepton^{(1)} \, \APnu^{(1)}_{\kern-0.10em \Plepton} \, \Pnu^{(1)}_{\kern-0.10em \Pgt}}\xspace}
\newcommand{\ellnunuTwo}{\ensuremath{\Plepton^{(2)} \, \APnu^{(2)}_{\kern-0.10em \Plepton} \, \Pnu^{(2)}_{\kern-0.10em \Pgt}}\xspace}
\newcommand{\ellMinusnunu}{\ensuremath{\Plepton^{-} \, \APnu_{\kern-0.10em \Plepton} \, \Pnu_{\kern-0.10em \Pgt}}\xspace}
\newcommand{\ellPlusnunu}{\ensuremath{\Plepton^{+} \, \Pnu_{\kern-0.10em \Plepton} \, \APnu_{\kern-0.10em \Pgt}}\xspace}
\newcommand{\enunu}{\ensuremath{\Pe \, \APnu_{\kern-0.10em \Pe} \, \Pnu_{\kern-0.10em \Pgt}}\xspace}
\newcommand{\mununu}{\ensuremath{\Pgm \, \APnu_{\kern-0.10em \Pgm} \, \Pnu_{\kern-0.10em \Pgt}}\xspace}
\newcommand{\nunu}{\ensuremath{\Pnu \, \APnu}\xspace}
\newcommand{\phat}{\ensuremath{\bm{\hat{p}}}\xspace}
\newcommand{\pT}{\ensuremath{p_{\textrm{T}}}\xspace}
\newcommand{\pThat}{\ensuremath{\hat{p}_{\textrm{T}}}\xspace}
\newcommand{\etahat}{\ensuremath{\hat{\eta}}\xspace}
\newcommand{\thetahat}{\ensuremath{\hat{\theta}}\xspace}
\newcommand{\phihat}{\ensuremath{\hat{\phi}}\xspace}
\newcommand{\mhat}{\ensuremath{\hat{m}}\xspace}
\newcommand{\Ehat}{\ensuremath{\hat{E}}\xspace}
\newcommand{\ET}{\ensuremath{E_{\textrm{T}}}\xspace}
\newcommand{\EThat}{\ensuremath{\hat{E}_{\textrm{T}}}\xspace}
\newcommand{\pX}{\ensuremath{p_{\textrm{x}}}\xspace}
\newcommand{\pXhat}{\ensuremath{\hat{p}_{\textrm{x}}}\xspace}
\newcommand{\pY}{\ensuremath{p_{\textrm{y}}}\xspace}
\newcommand{\pYhat}{\ensuremath{\hat{p}_{\textrm{y}}}\xspace}
\newcommand{\pZ}{\ensuremath{p_{\textrm{z}}}\xspace}
\newcommand{\pZhat}{\ensuremath{\hat{p}_{\textrm{z}}}\xspace}
\newcommand{\uX}{\ensuremath{u_{\textrm{x}}}\xspace}
\newcommand{\uY}{\ensuremath{u_{\textrm{y}}}\xspace}
\newcommand{\vX}{\ensuremath{v_{\textrm{x}}}\xspace}
\newcommand{\vY}{\ensuremath{v_{\textrm{y}}}\xspace}
\newcommand{\MET}{\ensuremath{p_{\textrm{T}}^{\textrm{\kern0.10em miss}}}\xspace}
\newcommand{\vecMET}{\ensuremath{\bm{p}_{\textrm{T}}^{\textrm{\kern0.10em miss}}}\xspace}
\newcommand{\METx}{\ensuremath{p_{\textrm{x}}^{\textrm{\kern0.10em miss}}}\xspace}
\newcommand{\METy}{\ensuremath{p_{\textrm{y}}^{\textrm{\kern0.10em miss}}}\xspace}
\newcommand{\MeV}{\ensuremath{\textrm{MeV}}\xspace}
\newcommand{\GeV}{\ensuremath{\textrm{GeV}}\xspace}
\newcommand{\TeV}{\ensuremath{\textrm{TeV}}\xspace}
\newcommand{\rec}{\ensuremath{\textrm{rec}}}
\newcommand{\true}{\ensuremath{\textrm{true}}}
\newcommand{\vis}{\ensuremath{\textrm{vis}}}
\newcommand{\inv}{\ensuremath{\textrm{inv}}}
\newcommand{\eff}{\ensuremath{\textrm{eff}}}
\newcommand{\miss}{\ensuremath{\textrm{miss}}}
\newcommand{\T}{\ensuremath{\textrm{T}}}
\newcommand{\cf}{cf.\xspace}
\newcommand{\ie}{i.e.\xspace}
\newcommand{\eg}{e.g.\xspace}
\newcommand{\BW}{\ensuremath{\textrm{BW}}}
\newcommand{\rad}{\ensuremath{\textrm{rad}}\xspace}
\newcommand{\mrad}{\ensuremath{\textrm{mrad}}\xspace}
\def\TReg{\textsuperscript{\textregistered}}
\usepackage{array}
\newcolumntype{C}[1]{>{\centering\arraybackslash}p{#1}}
%%%%%%%%%%%%%%%%%%%%%%%

\begin{document}

\ifx\ver\verPAPER
\begin{frontmatter}
\fi

\title{Reconstruction of the mass of Higgs boson pairs in events with Higgs boson pairs
  decaying into four $\Pgt$ leptons}

%% Group authors per affiliation:

\ifx\ver\verPreprint
%\author[1]{Lorenzo Bianchini}
%\author[2]{Betty Calpas}
%\author[3]{John Conway}
%\author[4]{Andrew Fowlie}
%\author[2, 5]{Luca Marzola}
%\author[2]{Lucia Perrini}
%\author[6]{Christian Veelken}
%\affil[1]{Institute for Particle Physics, ETH Zurich, 8093 Zurich, Switzerland}
%\affil[2]{National Institute for Chemical Physics and Biophysics, 10143 Tallinn, Estonia}
%\affil[3]{Department of Physics, University of California, Davis, CA 95616}
%\affil[4]{ARC Centre of Excellence for Particle Physics at the Tera-scale, Monash University, Melbourne, Victoria 3800, Australia}
%\affil[5]{Institute of Physics, University of Tartu, 50411 Tartu, Estonia}
%\affil[6]{CERN, 1211 Geneva, Switzerland}
\author[1]{Karl Ehat\"aht}
\author[1]{Luca Marzola}
\author[1]{Christian Veelken}
\affil[1]{National Institute for Chemical Physics and Biophysics, 10143 Tallinn, Estonia}
\fi
\ifx\ver\verPAPER
%\author[eth]{Lorenzo Bianchini}
%\ead{lorenzo.bianchini@cern.ch}
%\author[tallinn]{Betty Calpas}
%\ead{betty.calpas@cern.ch}
%\author[ucd]{John Conway}
%\ead{conway@physics.ucdavis.edu}
%\author[melb]{Andrew Fowlie}
%\ead{andrew.fowlie@monash.edu}
%\author[tartu]{Luca Marzola}
%\ead{luca.marzola@ut.ee}
%\author[tallinn]{Lucia Perrini}
%\ead{lucia.perrini@cern.ch}
%\author[cern]{Christian Veelken}
%\ead{christian.veelken@cern.ch}
%\address[eth]{Institute for Particle Physics, ETH Zurich, 8093 Zurich, Switzerland}
%\address[ucd]{Department of Physics, University of California, Davis, CA 95616}
%\address[tallinn]{National Institute for Chemical Physics and Biophysics, 10143 Tallinn, Estonia}
%\address[tartu]{Institute of Physics, University of Tartu, 51014 Tartu, Estonia}
%\address[melb]{ARC Centre of Excellence for Particle Physics at the Tera-scale, Monash University, Melbourne, Victoria 3800, Australia}
%\address[cern]{CERN, 1211 Geneva, Switzerland}
\author[tallinn]{Karl Ehat\"aht}
\ead{karl.ehataht@cern.ch}
\author[tartu]{Luca Marzola}
\ead{luca.marzola@ut.ee}
\author[tallinn]{Christian Veelken}
\ead{christian.veelken@cern.ch}
\address[tallinn]{National Institute for Chemical Physics and Biophysics, 10143 Tallinn, Estonia}
\fi

\ifx\ver\verPreprint
\maketitle
\fi

\begin{abstract}
Various theories beyond the Standard Model predict the existence of heavy resonances decaying to Higgs ($\PHiggs$) boson pairs.
In order to maximize the sensitivity of searches for such resonances, 
it is important that experimental analyses cover a variety of decay modes.
The decay of $\PHiggs$ boson pairs to four $\Pgt$ leptons ($\PHiggs\PHiggs \to \Pgt\Pgt\Pgt\Pgt$) has not been discussed in the literature so far.
This decay mode provides a small branching fraction, but also comparatively low backgrounds.
We present an algorithm for the reconstruction of the mass of the $\PHiggs$ boson pair in events in which the $\PHiggs$ boson pair
decays via $\PHiggs\PHiggs \to \Pgt\Pgt\Pgt\Pgt$ and the $\Pgt$ leptons subsequently decay into electrons, muons, or hadrons.
The algorithm achieves a resolution of $7$--$22\%$ relative to the mass of the $\PHiggs$ boson pair, 
depending on the mass of the resonance.
\end{abstract}

\ifx\ver\verPAPER
\end{frontmatter}
\fi

\clearpage

\linenumbers

%\begingroup
%\let\clearpage\relax
\include*{introduction}
\include*{algorithm}
\include*{performance}
\include*{summary}

\include*{appendix}

\bibliography{svFit4tau}
%\endgroup

\end{document}
