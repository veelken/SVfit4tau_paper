\section{The algorithm}
\label{sec:algorithm}

The reconstruction of $m_{\PHiggs\PHiggs}$ is based on maximizing the likelihood function:
\begin{align}
&
\mathcal{P}(\bm{p}^{\vis(1)},\bm{p}^{\vis(2)},\bm{p}^{\vis(3)},\bm{p}^{\vis(4)};\pX^{\rec},\pY^{\rec}|m_{\textrm{X}})
= \frac{32\pi^{4}}{s} \, \int \, d\Phi_{n} \, \cdot \hspace{2cm} \nonumber \\
& \qquad \vert \BW^{(1)}_{\Pgt} \vert^{2} \cdot \vert \mathcal{M}^{(1)}_{\Pgt\to\cdots}(\bm{\hat{p}}) \vert^{2} 
 \cdot \vert \BW^{(2)}_{\Pgt} \vert^{2} \cdot \vert \mathcal{M}^{(2)}_{\Pgt\to\cdots}(\bm{\hat{p}}) \vert^{2}
 \cdot \vert \BW^{(3)}_{\Pgt} \vert^{2} \cdot \vert \mathcal{M}^{(3)}_{\Pgt\to\cdots}(\bm{\hat{p}}}) \vert^{2}
 \cdot \vert \BW^{(4)}_{\Pgt} \vert^{2} \cdot \vert \mathcal{M}^{(4)}_{\Pgt\to\cdots}(\bm{\hat{p}}}) \vert^{2} \cdot \nonumber \\
& \qquad W(\bm{p}^{\vis(1)}|\bm{\hat{p}}^{\vis(1)}) \, W(\bm{p}^{\vis(2)}|\bm{\hat{p}}^{\vis(2)}) \, W(\bm{p}^{\vis(3)}|\bm{\hat{p}}^{\vis(3)}) \, W(\bm{p}^{\vis(4)}|\bm{\hat{p}}^{\vis(4)}) 
 \, W_{\rec}( \pX^{\rec},\pY^{\rec} | \pXhat^{\rec},\pYhat^{\rec} ) \, .
\label{eq:likelihood_with_hadRecoil}
\end{align}
with respect to the parameter $m_{\textrm{X}}$, 
the mass of the heavy particle $\textrm{X}$ that decays into a pair of $\PHiggs$ bosons.
We refer to the electron, muon, or hadrons produced in each $\Pgt$ decay as the ``visible'' $\Pgt$ decay products.
Their momentum is denoted by the symbol $\bm{p}^{\vis(i)$ ($i = 1,\ldots,4$).
Bold letters represent vector quantities.
The symbol $d\Phi_{n} = \prod_{i}^{n} \,
\frac{d^{3}\bm{p}^{(i)}}{(2\pi)^{3} \, 2 E_{(i)}}$ denotes the differential $n$-particle phase space element,
where $n$ refers to the number of particles in the final state.
The symbol $\vert \BW^{(i)}_{\Pgt} \vert^{2} \cdot \vert \mathcal{M}^{(i)}_{\Pgt\to\cdots}(\bm{\hat{p}}) \vert^{2}$ 
denotes the squared modulus of the ME for the decay of the $i$-th $\Pgt$ lepton.
The symbols $W(\bm{p}^{\vis(i)}|\bm{\hat{p}}^{\vis(i)})$ and $W_{\rec}( \pX^{\rec},\pY^{\rec} | \pXhat^{\rec},\pYhat^{\rec} )$ are referred to as ``transfer functions'' (TF).
They quantify the experimental resolutions with which the momenta of particles reconstructed in the event are measured.
More specifically, the TF $W(\bm{p}|\bm{\hat{p}})$ represents the probability density to observe the measured momentum $\bm{p}$,
given that its the true value of the momentum is $\bm{\hat{p}}$.
The symbol $W(\bm{p}^{\vis(i)}|\bm{\hat{p}}^{\vis(i)})$ represents the resolutions for measuring the momenta of the visible $\Pgt$ decay products,
while the symbol $W_{\rec}( \pX^{\rec},\pY^{\rec} | \pXhat^{\rec},\pYhat^{\rec} )$ quantifies the resolution for measuring the momentum, in the transverse plane, of the hadronic recoil.
The hadronic recoil is defined as the vectorial sum of all particles in the event that do not originate from the decay of the two $\PHiggs$ bosons.
The components $\pX^{\rec}$ and $\pY^{\rec}$ of its momentum are related to the corresponding components $\METx$ and $\METy$ of the ``missing transverse momentum'',
the vectorial sum of the momenta of all neutrinos produced in the $\Pgt$ lepton decays, via the relations:
\begin{align}
\METx = & -( \pX^{\rec} + \pX^{\vis(1)} + \pX^{\vis(2)} + \pX^{\vis(3)} + \pX^{\vis(4)} ) \nonumber \\
\METy = & -( \pY^{\rec} + \pY^{\vis(1)} + \pY^{\vis(2)} + \pY^{\vis(3)} + \pY^{\vis(4)} ) \nonumber \, .
\end{align}

The number of particles in the final state, $n$, depends on how many $\Pgt$ leptons decay to electrons or muons 
and how many decay to hadrons.
Following the formalism developed in Ref.~\cite{SVfitMEM}, 
we treat hadronic $\Pgt$ decays as two-body decays into a hadronic system $\tauh$ and a $\Pnut$.
Correspondingly, $n$ increases by $2$ for each leptonic ($\Pgt \to \enunu$ or $\Pgt \to \mununu$) and by $3$ units for each hadronic ($\Pgt \to \textrm{hadrons} + \Pnut$) $\Pgt$ decay.
Particles that are part of the hadronic recoil do not contribute to the differential $n$-particle phase space element $d\Phi_{n}$.
The dimension of the integral over the differential $n$-particle phase space element $d\Phi_{n}$ thus
varies between $8$ for events that decay via $\PHiggs\PHiggs \to \Pgt\Pgt\Pgt\Pgt \to \tauh\tauh\tauh\tauh$ 
and $12$ for events that decay via $\PHiggs\PHiggs \to \Pgt\Pgt\Pgt\Pgt \to \Plepton\Plepton\Plepton\Plepton$,
where the symbol $\Plepton$ refers to either an electron or a muon.

The dimensionality of the integration over the $n$-particle phase space element $d\Phi_{n}$ 
can be reduced by means of analytic transformations. 
Two (three) variables are sufficient to fully parametrize the kinematics of hadronic (leptonic) $\Pgt$ decays.
We choose to parametrize hadronic $\Pgt$ decays by the variables $z$ and $\phi_{\inv}$,
and leptonic $\Pgt$ decays by the variables $z$, $\phi_{\inv}$, and $m_{\inv}$.
The variable $z$ corresponds to the fraction of $\Pgt$ lepton energy, in the laboratory frame, that is carried by the visible $\Pgt$ decay products.
The variable $\phi_{\inv}$ specifies the orientation of the $\bm{p}^{\inv}$ vector relative to the $\bm{p}^{\vis}$ vector,
where the vector $\bm{p}^{\vis}$ refers to the momentum of the visible $\Pgt$ decay products,
and we denote by the vector $\bm{p}^{\inv}$ the momentum of the $\Pnut$ (the vectorial sum of the momenta of the two neutrinos) produced in hadronic (leptonic) $\Pgt$ decays.
The variable $m_{\inv}$ denotes the mass of the neutrino pair produced in leptonic $\Pgt$ decays.
Expressions for the product of the differential $n$-particle phase space element $d\Phi_{n}$ 
with the squared moduli $\vert \BW^{(i)}_{\Pgt} \vert^{2} \cdot \vert \mathcal{M}^{(i)}_{\Pgt\to\cdots}(\bm{\hat{p}}) \vert^{2}$ 
of the ME for the $\Pgt$ decays that are obtained by these analytic transformations are given by Eq.~[33] in Ref.~\cite{SVfitMEM}.
The expressions read:
\begin{align}
\vert \BW_{\Pgt} \vert^{2} \cdot \vert \mathcal{M}^{(i)}_{\Pgt\to\cdots}(\bm{\tilde{p}}) \vert^{2} \, d\Phi^{(i)}_{\tauhnu} 
 = & \, \frac{\pi}{m_{\Pgt}\Gamma_{\Pgt}} \,
 f_{\Phadron}\left(\bm{\hat{p}}^{\vis(i)}, m^{\vis(i)},
   \bm{\hat{p}}^{\inv(i)}\right) \, \frac{d^{3}\bm{\hat{p}}^{\vis}}{2 \hat{E}_{\vis}} \, dz \, d\phi_{\inv} \nonumber \\
\vert \BW_{\Pgt} \vert^{2} \cdot \vert \mathcal{M}^{(i)}_{\Pgt\to\cdots}(\bm{\tilde{p}}) \vert^{2} \, d\Phi^{(i)}_{\ellnunu} 
 = & \, \frac{\pi}{m_{\Pgt}\Gamma_{\Pgt}} \, f_{\ell}\left(\bm{\hat{p}}^{\vis(i)},
 m^{\vis(i)}, \bm{\hat{p}}^{\inv(i)}\right) \, \frac{d^{3}\bm{\hat{p}}^{\vis}}{2 \hat{E}_{\vis}} \, dz \, dm^{2}_{\inv} \, d\phi_{\inv}
 \nonumber \, ,
\end{align}
where the functions $f_{\Phadron}$ and $f_{\ell}$ given by:
\begin{align}
f_{h}\left(\bm{p}^{\vis}, m_{\vis}, \bm{p}^{\inv}\right) = &
  \frac{\vert\mathcal{M}^{\eff}_{\Pgt \to \tauh\Pnut}\vert^{2}}{256\pi^{6}} \cdot \frac{E_{\vis}}{\vert\bm{p}^{\vis}\vert \, z^{2}} \nonumber \\
f_{\Plepton}\left(\bm{p}^{\vis}, m_{\vis}, \bm{p}^{\inv}\right) = &
  \frac{I_{\inv}}{512\pi^{6}} \cdot \frac{E_{\vis}}{\vert\bm{p}^{\vis}\vert \, z^{2}} \nonumber \, . 
\end{align}

The knowledge that the four $\Pgt$ leptons originate from the decay of two $\PHiggs$ bosons is incorporated into the likelihood function 
$\mathcal{P}(\bm{p}^{\vis(1)},\bm{p}^{\vis(2)},\bm{p}^{\vis(3)},\bm{p}^{\vis(4)};\pX^{\rec},\pY^{\rec}|m_{\textrm{X}})$,
given by Eq.~\ref{eq:likelihood_with_hadRecoil}, by suitably chosen constraints.
For the purpose of defining the constraints, it is useful to enumerate the $\Pgt$ leptons 
such that the $\Pgt$ leptons with indices $i=1$ and $i=2$ are interpreted as originating from one and the same $\PHiggs$ boson.
We then demand that the visible $\Pgt$ decay products corresponding to the indices $i=1$ and $i=2$ have opposite charge,
and require the same for the visible $\Pgt$ decay products corresponding to the indices $i=3$ and $i=4$.
As the width of the $\PHiggs$ boson is known to be small~\cite{HIG-14-002,Aad:2015xua} compared to the experimental resolution that we aim to achieve on $m_{\PHiggs\PHiggs}$,
we choose to neglect it and use the narrow-width approximation (NWA) for each $\PHiggs$ boson.
The NWA introduces two $\delta$-functions, 
$\delta\left( (\hat{E}_{\vis(1)} + \hat{E}_{\inv(1)} + \hat{E}_{\vis(2)} + \hat{E}_{\inv(2)})^{2} - (\bm{\hat{p}}^{\vis(1)} + \bm{\hat{p}}^{\inv(1)} + \bm{\hat{p}}^{\vis(2)} + \bm{\hat{p}}^{\inv(2)})^{2} - m_{\PHiggs}^{2} \right)$ 
and 
$\delta\left( (\hat{E}_{\vis(3)} + \hat{E}_{\inv(3)} + \hat{E}_{\vis(4)} + \hat{E}_{\inv(4)})^{2} - (\bm{\hat{p}}^{\vis(3)} + \bm{\hat{p}}^{\inv(3)} + \bm{\hat{p}}^{\vis(4)} + \bm{\hat{p}}^{\inv(4)})^{2} - m_{\PHiggs}^{2} \right)$ 
into Eq.~\ref{eq:likelihood_with_hadRecoil}.
We make the simplifying assumption that the angle between the visible $\Pgt$ decay products and the neutrino is negligible for each $\Pgt$ lepton $i$,
\ie assume that the vectors $\bm{\hat{p}}^{\vis(i)$ and $\bm{\hat{p}}^{\inv(i)}$ are parallel.
The assumption is justified by the fact that $\pT$ of the visible $\Pgt$ decay products at the LHC are typically large compared to $m_{\Pgt} = 1.777$~\GeV~\cite{PDG}.
With this approximation, the $\delta$-functions simplify to 
$\delta\left(\frac{(\hat{E}_{\vis(1)} + \hat{E}_{\vis(2)})^{2} - (\bm{\hat{p}}^{\vis(1)} + \bm{\hat{p}}^{\vis(2)})^{2}}{z_{1} \, z_{2}} - m_{\PHiggs}^{2}\right)$ 
and 
$\delta\left(\frac{(\hat{E}_{\vis(3)} + \hat{E}_{\vis(4)})^{2} - (\bm{\hat{p}}^{\vis(3)} + \bm{\hat{p}}^{\vis(4)})^{2}}{z_{3} \, z_{4}} - m_{\PHiggs}^{2}\right)$.
They are used to eliminate the integration over the variables $z_{2}$ and $z_{4}$.
The $\delta$-function rule:
\begin{equation*} 
\delta \left( g(x) \right) = \sum_{k} \frac{\delta \left( x - x_{k}
  \right)}{\vert g'(x_{k}) \vert} 
\end{equation*}
yields the two factors $m_{\PHiggs}^{2}/z_{2}$ and $m_{\PHiggs}^{2}/z_{4}$,
with 
$z_{2} = \frac{(\hat{E}_{\vis(1)} + \hat{E}_{\vis(2)})^{2} - (\bm{\hat{p}}^{\vis(1)} + \bm{\hat{p}}^{\vis(2)})^{2}}{m_{\PHiggs}^{2} \, z_{1}}$ 
and 
$z_{4} = \frac{(\hat{E}_{\vis(3)} + \hat{E}_{\vis(4)})^{2} - (\bm{\hat{p}}^{\vis(3)} + \bm{\hat{p}}^{\vis(4)})^{2}}{m_{\PHiggs}^{2} \, z_{3}}$.

The evaluation of the remaining integral is performed in analogy with the procedure described in Section~3 of Ref.~\cite{SVfitMEM}.
Instead of computing the likelihood function for a series of values for the parameter $m_{\textrm{X}}$ and finding the value that maximizes the likelihood,
the following integral is computed:
\begin{align}
&
\mathcal{L}(\bm{p}^{\vis(1)},\bm{p}^{\vis(2)},\bm{p}^{\vis(3)},\bm{p}^{\vis(4)};\pX^{\rec},\pY^{\rec})
= \frac{32\pi^{4}}{s} \, \int \, dm_{\PHiggs} \, d\Phi_{n} \, \cdot \hspace{2cm} \nonumber \\
& \qquad \vert \BW^{(1)}_{\Pgt} \vert^{2} \cdot \vert \mathcal{M}^{(1)}_{\Pgt\to\cdots}(\bm{\hat{p}}) \vert^{2} 
 \cdot \vert \BW^{(2)}_{\Pgt} \vert^{2} \cdot \vert \mathcal{M}^{(2)}_{\Pgt\to\cdots}(\bm{\hat{p}}) \vert^{2}
 \cdot \vert \BW^{(3)}_{\Pgt} \vert^{2} \cdot \vert \mathcal{M}^{(3)}_{\Pgt\to\cdots}(\bm{\hat{p}}) \vert^{2}
 \cdot \vert \BW^{(4)}_{\Pgt} \vert^{2} \cdot \vert \mathcal{M}^{(4)}_{\Pgt\to\cdots}(\bm{\hat{p}}) \vert^{2} \cdot \nonumber \\
& \qquad W(\bm{p}^{\vis(1)}|\bm{\hat{p}}^{\vis(1)}) \, W(\bm{p}^{\vis(2)}|\bm{\hat{p}}^{\vis(2)}) \, W(\bm{p}^{\vis(3)}|\bm{\hat{p}}^{\vis(3)}) \, W(\bm{p}^{\vis(4)}|\bm{\hat{p}}^{\vis(4)}) 
 \, W_{\rec}( \pX^{\rec},\pY^{\rec} | \pXhat^{\rec},\pYhat^{\rec} ) \, \cdot \mathcal{F}(\bm{p}) \, ,
\label{eq:cSVfit_with_hadRecoil}
\end{align}
with 
$\mathcal{F}(\bm{p}) = \sqrt{(\hat{E}_{\vis(1)} + \hat{E}_{\inv(1)} + \hat{E}_{\vis(2)} + \hat{E}_{\inv(2)} + \hat{E}_{\vis(3)} + \hat{E}_{\inv(3)} + \hat{E}_{\vis(4)} + \hat{E}_{\inv(4)})^{2} - (\bm{\hat{p}}^{\vis(1)} + \bm{\hat{p}}^{\inv(1)} + \bm{\hat{p}}^{\vis(2)} + \bm{\hat{p}}^{\inv(2)} + \bm{\hat{p}}^{\vis(3)} + \bm{\hat{p}}^{\inv(3)} + \bm{\hat{p}}^{\vis(4)} + \bm{\hat{p}}^{\inv(4)})^{2}}$.
The evaluation of the integral is performed using a custom implementation of the Markov chain Monte Carlo integration method 
with the Metropolis--Hastings algorithm~\cite{Metropolis_Hastings}.
The estimate for the mass $m_{\textrm{X}}$ of the $\PHiggs$ boson pair is obtained by recording the value of $\mathcal{F}(\bm{p})$
for each evaluation of the integrand in Eq.~\ref{eq:cSVfit_with_hadRecoil} by the Markov chain and taking the median of the series of $\mathcal{F}(\bm{p})$ values.
The total number of evaluations of the integrand amounts to $100\,000$ per event. 
The first $10\,000$ evaluations are used as ``burn-in'' period to reach the equilibrium distribution of the Markov chain and are excluded from the computation of the median.

We note in passing that our implementation provides flexibility to choose different functions $\mathcal{F}(\bm{p})$, 
which allows to also reconstruct the $\pT$, $\eta$, and $\phi$ of the $\PHiggs$ boson pair,
while in this paper we will focus on the mass $m_{\PHiggs\PHiggs}$.

A remaining issue for the algorithm is that in $\textrm{X} \to \PHiggs\PHiggs \to \Pgt^{+}\Pgt^{-}\Pgt^{+}\Pgt^{-}$ events 
there exist two possibilities for building pairs of $\Pgt$ leptons of opposite charge.
The ambiguity is resolved, and a unique value of $m_{\PHiggs\PHiggs}$ is obtained for each event, 
by first discarding pairings for which the mass of the visible $\Pgt$ decay products exceeds $m_{\PHiggs}$
and then selecting the pairing for which the integrand in Eq.~\ref{eq:cSVfit_with_hadRecoil} attains the maximal value.
We will demonstrate in Section~\ref{sec:performance} that this choice yields the correct pairing in the vast majority of events.

