\section{The algorithm}
\label{sec:algorithm}

The reconstruction of the mass of the $\PHiggs$ boson pair is based on maximizing the likelihood function:
\begin{align}
& \mathcal{P}(\bm{p}^{\vis(1)},\bm{p}^{\vis(2)},\bm{p}^{\vis(3)},\bm{p}^{\vis(4)};\pX^{\rec},\pY^{\rec}|m_{\textrm{X}})
= \frac{32\pi^{4}}{s} \, \int \, d\Phi_{n} \, \cdot \nonumber \\
& \quad \delta\left( \left(\sum_{i=1}^{4} \, \hat{E}_{\Pgt(i)}\right)^{2} - \left(\sum_{i=1}^{4} \, \bm{\hat{p}}^{\Pgt(i)}\right)^{2} - m_{\textrm{X}} \right) 
  \delta\left( \pXhat^{\rec} + \sum_{i=1}^{4} \pXhat^{\Pgt(i)} \right) \cdot \delta\left( \pYhat^{\rec} + \sum_{i=1}^{4} \pYhat^{\Pgt(i)} \right) \cdot \nonumber \\
& \quad \vert \BW^{(1)}_{\Pgt} \vert^{2} \cdot \vert \mathcal{M}^{(1)}_{\Pgt\to\cdots}(\bm{\hat{p}}) \vert^{2} \cdot W(\bm{p}^{\vis(1)}|\bm{\hat{p}}^{\vis(1)}) \, \cdot 
\, \vert \BW^{(2)}_{\Pgt} \vert^{2} \cdot \vert \mathcal{M}^{(2)}_{\Pgt\to\cdots}(\bm{\hat{p}}) \vert^{2} \cdot W(\bm{p}^{\vis(2)}|\bm{\hat{p}}^{\vis(2)}) \, \cdot \nonumber \\
& \quad \vert \BW^{(3)}_{\Pgt} \vert^{2} \cdot \vert \mathcal{M}^{(3)}_{\Pgt\to\cdots}(\bm{\hat{p}}) \vert^{2} \cdot W(\bm{p}^{\vis(3)}|\bm{\hat{p}}^{\vis(3)}) \, \cdot
\, \vert \BW^{(4)}_{\Pgt} \vert^{2} \cdot \vert \mathcal{M}^{(4)}_{\Pgt\to\cdots}(\bm{\hat{p}}) \vert^{2} \cdot W(\bm{p}^{\vis(4)}|\bm{\hat{p}}^{\vis(4)}) \, \cdot \nonumber \\
& \quad W_{\rec}( \pX^{\rec},\pY^{\rec} | \pXhat^{\rec},\pYhat^{\rec} ) 
\label{eq:likelihood}
\end{align}
with respect to the parameter $m_{\textrm{X}}$, 
the mass of the postulated heavy particle $\textrm{X}$ that decays into a pair of $\PHiggs$ bosons.
We refer to the electron, muon, or hadrons produced in each $\Pgt$ decay as the ``visible'' $\Pgt$ decay products.
Their energy (momentum) is denoted by the symbol $E_{\vis(i)}$ ($\bm{p}^{\vis(i)}$), where the index $i$ ranges between $1$ and $4$.
The symbol $E_{\Pgt(i)}$ ($\bm{p}^{\Pgt(i)}$) denotes the energy (momentum) of the $i$-th $\Pgt$ lepton.
Bold letters represent vector quantities.
The true values of energies and momenta are indicated by a hat,
while symbols without a hat represent the measured values.
We use a Cartesian coordinate system, the $z$-axis of which is defined by the proton beam direction.
The symbol $d\Phi_{n} = \prod_{i}^{n} \,
\frac{d^{3}\bm{p}^{(i)}}{(2\pi)^{3} \, 2 E_{(i)}}$ denotes the differential $n$-particle phase-space element,
where $n$ refers to the number of particles in the final state.
The symbol $\vert \BW^{(i)}_{\Pgt} \vert^{2} \cdot \vert \mathcal{M}^{(i)}_{\Pgt\to\cdots}(\bm{\hat{p}}) \vert^{2}$ 
denotes the squared modulus of the ME for the decay of the $i$-th $\Pgt$ lepton.
The $\delta$-function 
$\delta\left( \left(\sum_{i=1}^{4} \, E_{\Pgt(i)}\right)^{2} - \left(\sum_{i=1}^{4} \, \bm{\hat{p}}^{\Pgt(i)}\right)^{2} - m_{\textrm{X}} \right)$
enforces the condition that the mass of the system of four $\Pgt$ leptons equals the value of the parameter $m_{\textrm{X}}$
given on the left-hand-side of the equation.

The functions $W(\bm{p}^{\vis(i)}|\bm{\hat{p}}^{\vis(i)})$ and $W_{\rec}( \pX^{\rec},\pY^{\rec} | \pXhat^{\rec},\pYhat^{\rec} )$ are referred to as ``transfer functions'' (TF).
They quantify the experimental resolutions with which the momenta of particles are measured in the detector.
The nomenclature $W(\bm{p}|\bm{\hat{p}})$ has the following meaning:
The value of the function $W(\bm{p}|\bm{\hat{p}})$ represents the probability density to observe the measured momentum $\bm{p}$,
given that the true value of the momentum is $\bm{\hat{p}}$.
The function $W(\bm{p}^{\vis(i)}|\bm{\hat{p}}^{\vis(i)})$ represents the resolution for measuring the momentum of the visible $\Pgt$ decay products,
while the function $W_{\rec}( \pX^{\rec},\pY^{\rec} | \pXhat^{\rec},\pYhat^{\rec} )$ quantifies the resolution for measuring the momentum, 
in the $x$-$y$ plane, of the hadronic recoil.

The hadronic recoil is defined as the vectorial sum of all particles in the event that do not originate from the decay of the two $\PHiggs$ bosons.
Conservation of momentum in the plane transverse to the beam direction implies that
the components $\pXhat^{\rec}$ and $\pYhat^{\rec}$ of its true momentum are equal to the negative sum of the momentum components
$\pXhat^{\Pgt(i)}$ and $\pYhat^{\Pgt(i)}$ of the four $\Pgt$ leptons,
\begin{equation*}
\pXhat^{\rec} = -\left( \sum_{i=1}^{4} \, \pXhat^{\Pgt(i)} \right) \quad \mbox{ and } \quad \pYhat^{\rec} = -\left( \sum_{i=1}^{4} \, \pYhat^{\Pgt(i)} \right) \, ,
\end{equation*}
as enforced by the two $\delta$-functions
$\delta\left( \pXhat^{\rec} + \sum_{i=1}^{4} \pXhat^{\Pgt(i)} \right)$ and $\delta\left( \pYhat^{\rec} + \sum_{i=1}^{4} \pYhat^{\Pgt(i)} \right)$
in the integrand.

The TF for the visible $\Pgt$ decay products and for the hadronic recoil are taken from Ref.~\cite{SVfitMEM}.
The resolution on the $\pT$ of $\tauh$ is modelled by the function:
\begin{equation}
W_{\Phadron}( \pT^{\vis} | \pThat^{\vis} ) = 
 \begin{cases}
   \mathcal{N} \, \xi_{1} \, \left( \frac{\alpha_{1}}{x_{1}} - x_{1} - \frac{x - \mu}{\sigma} \right)^{-\alpha_{1}} \,  
 & \mbox{if } x < x_{1} \\
   \mathcal{N} \, \exp\left( -\frac{1}{2} \, \left( \frac{x - \mu}{\sigma} \right)^{2} \right) \,
 & \mbox{if } x_{1} \leq x \leq x_{2} \\
   \mathcal{N} \, \xi_{2} \, \left( \frac{\alpha_{2}}{x_{2}} - x_{2} - \frac{x - \mu}{\sigma} \right)^{-\alpha_{2}} \, 
 & \mbox{if } x > x_{2} \, ,
 \end{cases}
\label{eq:tf_tauToHadDecays_pT}
\end{equation}
where we use the values $\mu = 1.0$, $\sigma = 0.03$, $x_{1} = 0.97$, $\alpha_{1} = 7$,
$x_{2} = 1.03$, and $\alpha_{2} = 3.5$ for its parameters,
while its $\eta$, $\phi$, and mass are assumed to be reconstructed with negligible experimental resolution.
The latter assumption is also made for the $\pT$, $\eta$, and $\phi$ of electrons and muons.
The momentum of the hadronic recoil is modelled by a two-dimensional normal distribution
and assumed to be reconstructed with a resolution of $\sigma_{\textrm{x}} = \sigma_{\textrm{y}} = 10$~\GeV 
on each of its components $\pX$ and $\pY$:
\begin{align}
W_{\rec}( \pX^{\rec},\pY^{\rec} | \pXhat^{\rec},\pYhat^{\rec} ) = & 
\frac{1}{2\pi \, \sqrt{\vert V \vert}} \, \exp \left( -\frac{1}{2}
  \left( \begin{array}{c} \Delta\pX^{\rec} \\ \Delta\pY^{\rec} \end{array} \right)^{T}
  \cdot V^{-1} \cdot
   \left( \begin{array}{c} \Delta\pX^{\rec} \\ \Delta\pY^{\rec} \end{array} \right)
  \right) \, , \nonumber \\
\quad \mbox{ with } \quad V = & \left( \begin{array}{cc} \sigma_{x}^{2} & 0 \\ 0 & \sigma_{y}^{2} \end{array} \right) \, .
\end{align}

The number of particles in the final state, $n$, depends on how many $\Pgt$ leptons decay to electrons or muons 
and how many decay to hadrons.
Following the formalism developed in Ref.~\cite{SVfitMEM}, 
we treat hadronic $\Pgt$ decays as two-body decays into a hadronic system $\tauh$ and a $\Pnut$.
Correspondingly, $n$ increases by $3$ for each $\Pgt$ lepton that decays to an electron or a muon
and by $2$ units for each $\Pgt$ lepton that decays hadronically.
Particles that are part of the hadronic recoil are treated as described in Section~2.2 of Ref.~\cite{SVfitMEM} 
and do not increase $n$.

The dimensionality of the integration over the phase-space element $d\Phi_{n}$ 
can be reduced by means of analytic transformations. 
Two (three) variables are sufficient to fully parametrize the kinematics of hadronic (leptonic) $\Pgt$ decays.
Following Ref.~\cite{SVfitMEM}, we choose to parametrize hadronic $\Pgt$ decays by the variables $z$ and $\phi_{\inv}$,
and leptonic $\Pgt$ decays by the variables $z$, $\phi_{\inv}$, and $m_{\inv}$.
The variable $z$ corresponds to the fraction of $\Pgt$ lepton energy, in the laboratory frame, that is carried by the visible $\Pgt$ decay products.
The variable $\phi_{\inv}$ specifies the orientation of the $\bm{p}^{\inv}$ vector relative to the $\bm{p}^{\vis}$ vector (see Fig.~\ref{fig:tauDecayParametrization} for illustration),
where the vector $\bm{p}^{\inv}$ refers to the vectorial sum of the momenta of the two neutrinos (to the momentum of the single $\Pnut$) produced in leptonic (hadronic) $\Pgt$ decays.
The variable $m_{\inv}$ denotes the mass of the neutrino pair produced in leptonic $\Pgt$ decays.

\begin{figure}[h]
\begin{center}
\includegraphics*[height=58mm]{figures/tauDecayParametrization.pdf}
\end{center}
\caption{
  Illustration of the variable $\phi_{\inv}$ that specifies the orientation of the $\bm{p}^{\inv}$ vector relative to the $\bm{p}^{\vis}$ vector.
  The angle $\theta_{\inv}$ between the $\bm{p}^{\inv}$ vector and the $\bm{p}^{\vis}$ vector is related to the variable $z$,
  as described in Section 2.4 of Ref.~\cite{SVfitMEM}, from which the illustration was taken.
}
\label{fig:tauDecayParametrization}
\end{figure} 

Expressions for the product of the phase-space element $d\Phi_{n}$ 
with the squared moduli $\vert \BW^{(i)}_{\Pgt} \vert^{2} \cdot \vert \mathcal{M}^{(i)}_{\Pgt\to\cdots}(\bm{\hat{p}}) \vert^{2}$ 
of the ME for the $\Pgt$ decays, 
obtained by the aforementioned transformations, are given by Eq.~(33) in Ref.~\cite{SVfitMEM}.
The expressions read:
\begin{align}
\vert \BW_{\Pgt} \vert^{2} \cdot \vert \mathcal{M}^{(i)}_{\Pgt\to\cdots}(\bm{\tilde{p}}) \vert^{2} \, d\Phi^{(i)}_{\tauhnu} 
 = & \, \frac{\pi}{m_{\Pgt} \, \Gamma_{\Pgt}} \,
 f_{\Phadron}\left(\bm{\hat{p}}^{\vis(i)}, m^{\vis(i)},
   \bm{\hat{p}}^{\inv(i)}\right) \, \frac{d^{3}\bm{\hat{p}}^{\vis}}{2 \hat{E}_{\vis}} \, dz \, d\phi_{\inv} \quad \mbox{ and } \nonumber \\
\vert \BW_{\Pgt} \vert^{2} \cdot \vert \mathcal{M}^{(i)}_{\Pgt\to\cdots}(\bm{\tilde{p}}) \vert^{2} \, d\Phi^{(i)}_{\ellnunu} 
 = & \, \frac{\pi}{m_{\Pgt} \, \Gamma_{\Pgt}} \, f_{\ell}\left(\bm{\hat{p}}^{\vis(i)},
 m^{\vis(i)}, \bm{\hat{p}}^{\inv(i)}\right) \, \frac{d^{3}\bm{\hat{p}}^{\vis}}{2 \hat{E}_{\vis}} \, dz \, dm^{2}_{\inv} \, d\phi_{\inv}
 \nonumber \, ,
\end{align}
where the functions $f_{\Phadron}$ and $f_{\ell}$ are defined as:
\begin{align}
f_{h}\left(\bm{p}^{\vis}, m_{\vis}, \bm{p}^{\inv}\right) = &
  \frac{\vert\mathcal{M}^{\eff}_{\Pgt \to \tauh\Pnut}\vert^{2}}{256\pi^{6}} \cdot \frac{E_{\vis}}{\vert\bm{p}^{\vis}\vert \, z^{2}} \quad \mbox{ and } \nonumber \\
f_{\Plepton}\left(\bm{p}^{\vis}, m_{\vis}, \bm{p}^{\inv}\right) = &
  \frac{I_{\inv}}{512\pi^{6}} \cdot \frac{E_{\vis}}{\vert\bm{p}^{\vis}\vert \, z^{2}} \nonumber \, , 
\end{align}
with:
\begin{align}
\vert \mathcal{M}^{\textrm{eff}}_{\Pgt \to \tauhnu} \vert^{2} = & 16 \pi \, m_{\Pgt} \, \Gamma_{\Pgt} \cdot
  \frac{m_{\Pgt}^{2}}{m_{\Pgt}^{2} - m_{\vis}^{2}} \cdot \mathcal{B}(\Pgt \to \textrm{hadrons} + \Pnut) \quad \mbox { and } \nonumber \\
I_{\inv} = & \, \frac{1}{2} \, m_{\inv} \, \int \, \frac{d\Omega_{v}}{(2\pi)^{3}} \, 
  \vert\mathcal{M}_{\Pgt \to \ellnunu}\vert^{2} \, , \quad \mbox { where } \nonumber \\ 
\vert\mathcal{M}_{\Pgt \to \ellnunu} \vert^{2} = & 64 \, G^{2}_{F} \,
  \left( E_{\Pgt} \, E_{\APnu_{\Plepton}} - \bm{p}^{\Pgt} \cdot
  \bm{p}^{\APnu_{\Plepton}} \right) \, \left( E_{\Plepton} \,
  E_{\Pnut} - \bm{p}^{\Plepton} \cdot \bm{p}^{\Pnut} \right) \nonumber 
\end{align}
and $\mathcal{B}(\Pgt \to \textrm{hadrons} + \Pnut) = 0.648$~\cite{PDG} 
denotes the measured branching fraction for $\Pgt$ leptons to decay hadronically.

The knowledge that the four $\Pgt$ leptons originate from the decay of two $\PHiggs$ bosons is incorporated into the likelihood function 
$\mathcal{P}$ by suitably chosen constraints.
For the purpose of defining the constraints, it is useful to enumerate the $\Pgt$ leptons 
such that the two $\Pgt$ leptons with indices $i=1$ and $i=2$ (and similarly the two $\Pgt$ leptons with indices $i=3$ and $i=4$) are interpreted as originating from the same $\PHiggs$ boson.
We then require that the visible $\Pgt$ decay products corresponding to the indices $i=1$ and $i=2$ have opposite charge,
and the same applies to the visible $\Pgt$ decay products corresponding to the indices $i=3$ and $i=4$.
As the width of the $\PHiggs$ boson is known to be small~\cite{HIG-14-002,Aad:2015xua} compared to the experimental resolution that we aim to achieve on $m_{\PHiggs\PHiggs}$,
we choose to neglect it and use the narrow-width approximation (NWA) for each $\PHiggs$ boson.
The NWA introduces two $\delta$-functions, 
\begin{align}
 & \delta\left( (\hat{E}_{\Pgt(1)} + \hat{E}_{\Pgt(2)})^{2} - (\bm{\hat{p}}^{\Pgt(1)} + \bm{\hat{p}}^{\Pgt(2)})^{2} - m_{\PHiggs}^{2} \right) \quad \mbox{ and } \nonumber \\
 & \delta\left( (\hat{E}_{\Pgt(3)} + \hat{E}_{\Pgt(4)})^{2} - (\bm{\hat{p}}^{\Pgt(3)} + \bm{\hat{p}}^{\Pgt(4)})^{2} - m_{\PHiggs}^{2} \right)
\end{align}
into Eq.~(\ref{eq:likelihood}).
For the purpose of evaluating the $\delta$-functions,
we make the simplifying assumption that the angle between the vectors $\bm{\hat{p}}^{\vis(i)}$ and $\bm{\hat{p}}^{\inv(i)}$ is negligible.
The assumption is justified by the fact that at the LHC the $\pT$ of the visible $\Pgt$ decay products are typically large compared to the mass, 
$m_{\Pgt} = 1.777$~\GeV~\cite{PDG}, of the $\Pgt$ lepton.
With this assumption, the $\delta$-functions simplify to:
\begin{equation*}
\delta\left(\frac{m_{\vis(12)}}{z_{1} \, z_{2}} - m_{\PHiggs}^{2}\right) \quad \mbox{ and } \quad \delta\left(\frac{m_{\vis(34)}}{z_{3} \, z_{4}} - m_{\PHiggs}^{2}\right) \, ,
\end{equation*}
where we denote by the symbol $m_{\vis(ij)}$ the ``visible mass'' of the decay products of $\Pgt$ leptons $i$ and $j$:
\begin{equation*}
m_{\vis(ij)} = (\hat{E}_{\vis(i)} + \hat{E}_{\vis(i)})^{2} - (\bm{\hat{p}}^{\vis(i)} + \bm{\hat{p}}^{\vis(j)})^{2} \, .
\end{equation*}
The $\delta$-functions are used to eliminate the integration over the variables $z_{2}$ and $z_{4}$.
The $\delta$-function rule,
\begin{equation*} 
\delta \left( g(x) \right) = \sum_{k} \, \frac{\delta \left( x - x_{k}
  \right)}{\vert g'(x_{k}) \vert} \, ,
\end{equation*}
where the sum extends over all roots $x_{k}$ of the function $g(x)$,
yields the two factors:
\begin{equation*}
\frac{z_{2}}{m_{\PHiggs}^{2}} \quad \mbox{ and } \quad \frac{z_{4}}{m_{\PHiggs}^{2}} \, ,
\end{equation*}
with the roots:
\begin{equation*}
z_{2} = \frac{m_{\vis(12)}}{m_{\PHiggs}^{2} \, z_{1}} \quad \mbox{ and } \quad z_{4} = \frac{m_{\vis(34)}}{m_{\PHiggs}^{2} \, z_{3}} \, .
\end{equation*}

The condition
$\delta\left( \left(\sum_{i=1}^{4} \, E_{\Pgt(i)}\right)^{2} - \left(\sum_{i=1}^{4} \, \bm{\hat{p}}^{\Pgt(i)}\right)^{2} - m_{\textrm{X}} \right)$
is used to eliminate the integration over the variable $z_{3}$.
It yields the factor:
\begin{equation}
\lvert \frac{z_{1} \, z_{3}^{2}}{b \, z_{3}^{2} - c} \rvert \, ,
\label{eq:deltaFuncFactor}
\end{equation}
with the two roots:
\begin{equation*}
z_{3}^{(+)} = \frac{a + \sqrt{b}}{c} \quad \mbox{ and } \quad z_{3}^{(-)} = \frac{a - \sqrt{b}}{c} \, ,
\end{equation*}
where:
\begin{align}
a = & (m_{\textrm{X}}^{2} - 2 \, m_{\PHiggs}^{2}) \, z_{1} \, , \nonumber \\
b = & \frac{m_{\vis(14)}^{2}}{m_{\vis(34)}^{2}} \, m_{\PHiggs}^{2} + \frac{m_{\vis(24)}^{2}}{m_{\vis(12)}^{2} \, m_{\vis(34)}^{2}} \, m_{\PHiggs}^{4} \, z_{1}^{2} \quad \mbox{ and } \nonumber \\
c = & m_{\vis(13)}^{2} + \frac{m_{\vis(23)}^{2}}{m_{\vis(12)}^{2}} \, z_{1}^{2} \, .
\end{align}
The requirement that the energies of electrons, muons, and $\tauh$ 
as well as the energies of the neutrinos produced in the $\Pgt$ decays are positive
restricts the variable $z_{3}$ to the range $0 < z_{3} \leq 1$.
In case the roots $z_{3}^{(+)}$ and $z_{3}^{(-)}$ are both within this range,
the integrand is evaluated for each root separately and the values obtained for each root are summed.
Otherwise, only the root satisfying the condition $0 < z_{3} \leq 1$ is retained.

Expressions for the likelihood function $\mathcal{P}$, obtained after performing these analytic transformations, 
are given by Eqs.~(\ref{eq:likelihood_thththth}) to~(\ref{eq:likelihood_llll}) in the Appendix.
We refer to the different decay channels of the four $\Pgt$ leptons as 
$\tauh\tauh\tauh\tauh$, $\Plepton\tauh\tauh\tauh$, $\Plepton\Plepton\tauh\tauh$, $\Plepton\Plepton\Plepton\tauh$, and $\Plepton\Plepton\Plepton\Plepton$, 
where the symbol $\Plepton$ refers to an electron or muon,
and the neutrinos produced in the $\Pgt$ decays are omitted from the nomenclature.
The dimension of integration varies between $5$ for events in the $\tauh\tauh\tauh\tauh$ decay channel and $9$ for events in the $\Plepton\Plepton\Plepton\Plepton$ channel.
The expressions given in the Appendix correspond to one particular association of reconstructed electrons, muons, and $\tauh$ 
to the indices $1$, $2$, $3$, and $4$, which enumerate the $\Pgt$ decay products in Eqs.~(\ref{eq:likelihood_thththth}) to~(\ref{eq:likelihood_llll}).
Expressions for alternative associations can be obtained by appropriate permutations of the indices.

For any one of these associations
the best estimate, $m_{\PHiggs\PHiggs}$, for the mass of the $\PHiggs$ boson pair is obtained 
by finding the value of $m_{\textrm{X}}$ that maximizes the value of $\mathcal{P}$.
The integrand in Eqs.~(\ref{eq:likelihood_thththth}) to~(\ref{eq:likelihood_llll})
is evaluated for a series of mass hypotheses $m_{\textrm{X}}^{(i)}$.
Starting from the initial value $m_{\textrm{X}}^{(0)} = 1.0125 \cdot \max (2 \, m_{\PHiggs}, m_{\PHiggs\PHiggs}^{\vis})$,
where
$m_{\PHiggs\PHiggs}^{\vis} = \sqrt{\left(\sum_{i=1}^{4} \, E_{\vis(i)}\right)^{2} - \left(\sum_{i=1}^{4} \, \bm{p}^{\vis(i)}\right)^{2}}$,
the next mass hypothesis in the series is defined by the recursive relation $m_{\textrm{X}}^{(i+1)} = (1 + \delta) \cdot m_{\textrm{X}}^{(i)}$.
The step size $\delta = 0.025$ is chosen such that it is small compared to the resolution on $m_{\PHiggs\PHiggs}$
that we expect our algorithm to achieve.
The evaluation of the integral is performed numerically, using the VAMP algorithm~\cite{VAMP},
an improved implementation of the VEGAS algorithm~\cite{VEGAS}.
For each mass hypothesis $m_{\textrm{X}}^{(i)}$, the integrand is evaluated $20\,000$ times.

We note in passing that our algorithm alternatively supports an integration methods based on a custom implementation
of the Markov-Chain integration method with the Metropolis--Hastings algorithm~\cite{Metropolis_Hastings}.
The latter allows to reconstruct the $\pT$, pseudo-rapidity $\eta$, and azimuthal angle $\phi$ of the resonance $\textrm{X}$ also.
In this paper, we focus on the reconstruction of the mass, however.

A remaining issue for the algorithm is that in $\PHiggs\PHiggs \to \Pgt\Pgt\Pgt\Pgt$ events 
there exist two possibilities for building pairs of $\Pgt$ leptons of opposite charge.
The ambiguity is resolved, and a unique value of $m_{\PHiggs\PHiggs}$ is obtained for each event, 
by first discarding pairings for which either $m_{\vis(12)}$ or $m_{\vis(34)}$ exceeds $m_{\PHiggs}$
and then selecting the pairing for which the likelihood function $\mathcal{P}$
attains the maximal value (for any $m_{\textrm{X}}$).
We will demonstrate in Section~\ref{sec:performance} that this choice yields the correct pairing for the majority of events.





